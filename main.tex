\documentclass{myclass}
\usepackage{lipsum} 

\begin{document}

% 1. 生成封面
% 确保文件夹里放了一张名为 logo.png (或 logo.jpg) 的图片,否则会报错
\makecover 

\Englishmakecover
% 2. 正文部分测试
\setcounter{page}{1} % 重置页码
\pagenumbering{Roman} % 使用罗马数字页码

\begin{cnabstract}

摘要是论文(设计)内容不加注释和评论的简短陈述,应具有独立性和自明
性,即不阅读论文(设计)的全文,就可以获得必要的信息。

摘要一般应说明研究工作的目的 和意义、研究思想和方法、研究过程、研究
结果和最终结论等。摘要中一般不用图、表、化学结构式、计算机程序,不用非
公知公用的符号、术语和非法定的计量单位。

中文摘要一般为 300~500 汉字。

摘要页置于英文题名页后。

关键词是从论文(设计)题名、摘要或正文中选取的对表示论文(设计)主
题内容起关键作用,且具有检索意义的词或词组。一般每篇论文(设计)应选取
3~5 个词作为关键词,以显著的字符另起一行,排在同种语言摘要的下方,尽量用
《汉语主题词表》或各专业主题词表提供的规范词。

关键词与摘要的内容之间空一行。关键词的词间用分号间隔,末尾不加标点。

\end{cnabstract}

\cnkeywords{关键词一;关键词二;关键词三}

% 英文摘要
\begin{enabstract}
This is the English abstract. The abstract should be a concise summary of the thesis (design), providing an overview of the research purpose, methodology, process, results, and conclusions. It should be self-contained and independent, allowing readers to grasp the essential information without reading the full text.

Figures, tables, chemical structural formulas, computer programs, non-public symbols, terms, and non-statutory measurement units should generally not be used in the abstract.

The English abstract is typically 300-500 words and is placed after the English title page.
\end{enabstract}
\enkeywords{Keyword One; Keyword Two; Keyword Three}


\tableofcontents

% 5. 正文部分(需要手动添加目录条目)

\setcounter{page}{1} % 重置页码
\pagenumbering{arabic} % 使用阿拉伯数字页码

\section{荷载算}
\subsection{研究背景}
这里是绪论内容...

\subsubsection{哈哈哈哈}

\subsubsection{测试测试}

\subsection{研究意义}
研究意义的内容...

\section{文献综述}
\subsection{国内外研究现状}
文献综述内容...

\subsection{研究评述}
研究评述内容...

\section{研究方法}
\subsection{研究方法设计}
方法设计内容...

\subsection{实验方案}
实验方案内容...

\section{研究结果与分析}
\subsection{实验结果}
结果内容...

\subsection{结果分析}
分析内容...

\section{结论与展望}
\subsection{研究结论}
结论内容...

\subsection{研究展望}
展望内容...


\end{document}